\chapter{Podsumowanie}\label{chap:introduction}
W ramach pracy stworzono aplikację wspierającą treningi biegowe. Użytkownicy z niej korzystający mają możliwość zapisywania zarówno przebiegu tras jak i cech, które je charakteryzują. Zapisane trasy mogą być wyszukiwane przez społeczność biegaczy na podstawie wprowadzonych kryteriów wyszukiwania, które są ściśle powiązane z określonymi cechami. Dodatkowo możliwe jest wzięcie udziału w procesie wirtualnej rywalizacji, co potęguje odczucia i satysfakcję płynącą z wykonanego treningu. 

W rozdziale \ref{chap:istniejace} zawarto analizę funkcjonalności aplikacji przeznaczonych do treningu biegowego, które cieszą się wysoką popularnością i mocną pozycją na rynku. Na jej podstawie można stwierdzić, że w przeciwieństwie do aplikacji stworzonej w ramach niniejszej pracy, żadna z nich nie pozwala na zapisywanie tras ze szczegółowo określonymi cechami, wyszukiwanie ich oraz przeprowadzanie procesu wirtualnej rywalizacji.

Oznacza to, że udało się spełnić wszystkie cele pracy wymienione w rozdziale \ref{chap:cele-pracy}, a najważniejszym wnioskiem płynącym z niniejszej pracy jest fakt, iż udało się stworzyć aplikację przeznaczoną do treningów biegowych o unikatowej funkcjonalności.